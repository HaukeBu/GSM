\documentclass[11pt]{article}
\usepackage{graphicx}
\usepackage{hyperref} 
\usepackage[figure]{hypcap}
%Gummi|065|=)

\title{\textbf{Interoperable IoT Platform Concept}}
\author{Hauke Buhr}
\date{20.04.2017 \\ Version 0.6}
\begin{document}

\maketitle

\section{Abstract}
% Introduction
The Internet of Things(IoT) is currently a fast developing topic. It takes place in areas like smart home, smart city, health care and many
more which could be found in\footnote[3]{KaaProject, IoT Use Cases, https://www.kaaproject.org/iot-use-cases/ , April the 13th 2017}.
Analysts are speaking of 50 billion IoT devices by 2022\footnote[1]{Hoan-Suk Choi, Deok-Hee Kang and Woo-Seop Rhee, “RISE: Role-based
Internet of Things Service Environment”, Hanbat National University, Daejeon/Korea, IEEE, 2016.}. This tremendous amount of devices is generating even bigger amounts of data. The produced data has to be processed in a efficient way to be able to compute the data in a proper time. \\

% Problem scenario
Current open source IoT platforms are vertical designed and cover everything from sensor platform SDKs up to data evaluation and
presentation services. But they do not cover interoperability between itself and any other platform. That means the data is stored and
accessible inside the system but could not be accessed by other platforms by default in a convenient way. There are also approaches
heading to an interoperable platform design. Mostly they are focusing on platform intercommunication and not on a interoperable discovery of
IoT objects(further also mentioned as things) and data.
\\

\begin{figure}[ht]
	\centering
  \includegraphics[scale=0.4]{../pictures/Toplevel_no_notes.jpg}
  	\caption{Toplevel View}
	\label{fig:toplevel}
\end{figure}
% Suggested Solution
This bachelor thesis proposes an interoperable IoT platform concept which is interoperable by design. The concept consists of two types of
services, first the core services and second connected services.
The proposed core services are management-, registry- and gateway services. Also several interface specifications are introduced, as you
can see in \ref{fig:toplevel}. Connected services are here mentioned as \textit{things}, upper level services like data processing and data
bases or presenting services. Furthermore registries can share their dictionaries with other registries. Shared dictionaries and the
resulting platform are leading to a stacked IoT platform which could be accessed at any level.
This approach counteracts a closed and monolithic platform design.
\\

Furthermore an interoperable platform needs a \textit{thing} discovery. The
proposed discovery mechanism needs a generic \textit{thing} description scheme. The proposed description scheme is an object oriented
description scheme.
Also this thesis describes the results which were made with a proof of concept simulation in OMNeT++. The \textit{thing} description scheme
considers several features which are reducing the messages(proposed by Choi, Kang and Rhee\footnotemark[1]), makes different data priorities(proposed
by Choi, Kang and Rhee\footnotemark[1]) and interaction with IoT \textit{things} in three different modes (proposed by Vandikas and
Tsiatsis\footnote[2]{Konstantinos Vandikas and Vlasios Tsiatsis, “Performance Evaluation of an IOT Platform”, Ericsson Research,
Stockholm/Sweden, IEEE computer society, 2014.}) possible. \\

The first step is to design the \textit{thing} description scheme and a corresponding test framework. The second step is the
platform design itself and the third step is the communication design based on proposals of the papers\footnotemark[1] \footnotemark[2]. \\

%The results of this thesis have also been made available on the following GitHub repository: \emph{http://www.github.com/...}

\paragraph{Keywords}: Internet of Things(IoT), Interoperable Core IoT Platform, Distributed platform, Thing design scheme, Communication
design, OMNeT++
\end{document}