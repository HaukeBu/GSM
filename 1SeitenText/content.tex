Das Internet der Dinge~(IoT) ist im Moment ein schnell wachsendes Themengebiet.
Es findet Anwendung in vielen verschiedenen Bereichen wie zum Beispiel dem Smart Home, der Smart City, dem Gesundheitswesen~\cite{kaa}.

Analysten prognostizieren, dass 50 Milliarden IoT Geräte im Jahr 2022 im Umlauf sein werden~\cite{rise}. 
Es ist zu erwarten, dass diese große Anzahl von Geräten erzeugt eine sehr große Menge an Daten. 
Die generierten Daten müssen daher effizient verwaltet werden, um die Daten in angemessener Zeit verarbeiten und übertragen zu können. \newline

% Problem Szenario
\noindent Gegenwärtige Open-Source IoT Plattformen sind vertikal entworfen und decken die IoT Geräte Hersteller SDKs sowie auch Datenauswertungs- und Darstellungsdiensten ab. 
Indes decken gegenwärtige Plattformen nicht die Interoperabilität zwischen der eigenen und anderen Plattformen ab. 
Aus der fehlenden Interoperabilität kann gefolgert werden, dass die Daten nur  innerhalb der eigenen Plattform gespeichert und verwertbar sind.
Der Zugriff auf die Daten wird so jedoch nicht problemlos von außerhalb möglich. 

Es gibt auch IoT Plattformen, die einen Ansatz zur Interoperabilität enthalten. 
Doch meist konzentrieren sich die interoperablen Plattformen nur darauf, die Kommunikation von IoT Geräten untereinander zu gewährleisten. \newline

% Suggested Solution
%% Nochmal überarbeiten ! Objecte als Dinge!
\noindent Diese Bachelorthesis schlägt ein interoperables IoT Plattformkonzept vor, welches durch sein Design die Eigenschaft der Interoperabilität unterstützt. 
Das Konzept besteht aus zwei Strukturelementen, zum einen die Kernkomponenten und zum anderen die \glqq Connected Services\grqq{}. 
Die Kernkomponenten sind die Management-, Registry- und Gatewaykomponente. 
Außerdem werden mehrere Interfaces für eine einheitliche Schnittstellenbeschreibung eingeführt.
\glqq Connected Services\grqq{} werden im Folgenden als \glqq Objekte\grqq{} \footnote[1]{Objekte nicht im Kontext einer objektorientierten Programmiersprache.} bezeichnet. 
Diese Objekte sind \glqq Services\grqq{}\footnote[2]{Services als angebotene Dienste, die Daten von Objekten nutzen.} auf einer höheren Ebene, die Daten verarbeiten, speichern und, oder diese präsentieren. 

Die in dieser Thesis vorgestellte Registry kann ihre Einträge mit einer anderen Registry teilen. 
Diese verteilten Registryeinträge und die angestrebte Plattform führen zu einer Hierarchie der IoT Plattformen, deren Daten auf jeder freigegebenen Ebene verfügbar sind. 
Dieser Ansatz fördert somit einem heterogenen und offenen Plattformdesign.

Außerdem braucht eine interoperable Plattform eine Objektsuchfunktion. 
Die vorgestellte Suchfunktion benötigt eine generische Beschreibungssprache. 
Die in dieser Thesis vorgestellte Beschreibungssprache ist objektorientiert.
 
Zu dem werden in dieser Thesis die Ergebnisse aus einer Proof of Concept Simulation in OMNeT++~\cite{omnetpp_simulation_manual} bereitgestellt und ausgewertet. 
Es zeigt sich, dass die Komplexität der Objektbeschreibung minimiert wird, da nur wenige Merkmale beschrieben werden müssen. 
Jedoch besteht durch den erweiterbaren Aufbau der Objektbeschreibung eine effektivere Möglichkeit der Beschreibung.

Auch die vorgeschlagenen Kommunikationsmuster in dieser Thesis sollen das Nachrichtenaufkommen reduzieren und eine Priorisierung von Nachrichten ermöglichen~\cite{rise}.\newline

\noindent Teil dieser Bachelorthesis ist die Analyse von Anforderungen an eine IoT Plattform, mit anschließender Analyse einiger bestehender Plattformen.
Des Weiteren werden Kommunikationsmuster definiert und die Objektbeschreibungssprache wird designt.
Am Ende wird ein Entwurf der Plattform angeboten und mit OMNet++ demonstriert und evaluiert.\newline

%The results of this thesis have also been made available on the following GitHub repository: \emph{http://www.github.com/...}